\documentclass{article}

\usepackage{amsmath, amsthm, amssymb, amsfonts}
\usepackage{thmtools}
\usepackage{graphicx}
\usepackage{setspace}
\usepackage{geometry}
\usepackage{float}
\usepackage{hyperref}
\usepackage[utf8]{inputenc}
\usepackage[english]{babel}
\usepackage{framed}
\usepackage[dvipsnames]{xcolor}
\usepackage{tcolorbox}

\colorlet{LightGray}{White!90!Periwinkle}
\colorlet{LightOrange}{Orange!15}
\colorlet{LightGreen}{Green!15}
\colorlet{LightBlue}{Blue!15}

\newcommand{\HRule}[1]{\rule{\linewidth}{#1}}

\declaretheoremstyle[name=Theorem,]{thmsty}
\declaretheorem[style=thmsty,numberwithin=section]{theorem}
\tcolorboxenvironment{theorem}{colback=LightGray}

\declaretheoremstyle[name=Proposition,]{prosty}
\declaretheorem[style=prosty,numberlike=theorem]{proposition}
\tcolorboxenvironment{proposition}{colback=LightOrange}

\declaretheoremstyle[name=Principle,]{prcpsty}
\declaretheorem[style=prcpsty,numberlike=theorem]{principle}
\tcolorboxenvironment{principle}{colback=LightGreen}

\declaretheoremstyle[name=Definition,]{defsty}
\declaretheorem[style=defsty,numberlike=theorem]{definition}
\tcolorboxenvironment{definition}{colback=LightBlue}

\declaretheoremstyle[name=Lemma,]{lemsty}
\declaretheorem[style=lemsty,numberlike=theorem]{lemma}
\tcolorboxenvironment{lemma}{colback=LightGray}

\setstretch{1.2}
\geometry{
    textheight=9in,
    textwidth=5.5in,
    top=1in,
    headheight=12pt,
    headsep=25pt,
    footskip=30pt
}

\setlength{\parindent}{0pt}

% ------------------------------------------------------------------------------
% Helper commands
% ------------------------------------------------------------------------------

\newcommand{\SecHead}[1]{\large \textbf{#1} \\}
\newcommand{\bb}[1]{\mathbb{#1}}

% ------------------------------------------------------------------------------

\begin{document}

% ------------------------------------------------------------------------------
% Cover Page and ToC
% ------------------------------------------------------------------------------

\title{ \normalsize \textsc{}
\\ [2.0cm]
\HRule{1.5pt} \\
\LARGE \textbf{\uppercase{Differential Geometry}
\HRule{2.0pt} \\ [0.6cm] \LARGE{Spring 2024} \vspace*{10\baselineskip}}
}
\date{}
\author{\textbf{Akshat Adsule}}

\maketitle
\newpage

\hypersetup{hidelinks}
\tableofcontents
\newpage

% ------------------------------------------------------------------------------

\section{Review of Vector Analysis \& Linear Algebra}
\SecHead{Vector Spaces}
A vector space is a set $V$ whose elements are vectors for which
the two binary operations $+ : V \times V \rightarrow V$ and $\cdot : R \times V \rightarrow V$ are define as so.
Vector spaces must also satisfy the following axioms for all $u, v, w \in V$ and $r, s \in R$.
\begin{itemize}
  \item $u + v = v + u$
  \item $u + (v + w) = (u + v) + w$
  \item $\exists \ 0 \in V $ s.t. $ 0 + u = u$
  \item $(rs) \cdot u = r\cdot(s \cdot u)$
  \item $(r + s) \cdot u = r\cdot u + s \cdot u$
  \item $r \cdot (u + v) = r\cdot u + r \cdot v$
  \item $0 \cdot u = 0$
  \item $1 \cdot u = u$
\end{itemize}
We also define some properties of vector spaces:

\begin{definition} [Linearly Independence] \ \\
  A set $\{v_i | i \in I\}$ is \textbf{linearly independent} if whether
  a finite linear combination $\sum a^i v_i = 0$, then each $a^i = 0$.
  If it is possible to find a finite linear combination that is not zero,
  then the set is \textbf{linearly dependent}.
\end{definition}

\begin{definition} [Span] \ \\
  A subset $S \subset V$ \textbf{spans} $V$ if for each vector $v \in V$
  there are vectors $v_1, v_2, \dots, v_n \in S$ and real numbers $a^1, a^2, \dots, a^n$
  such that $v = \sum a^i v_i$.
\end{definition}

\begin{definition} [Basis] \ \\
  A \textbf{basis} for a vector space $V$ is a linearly independent set
  of vectors that spans $V$.
\end{definition}

\begin{definition} [Dimension] \ \\
  The \textbf{dimension} of a vector space $V$ is the number of vectors in a basis for $V$.
\end{definition}

With this, we can introduce the following theorem:

\begin{theorem} \ \\
  If $V$ is a vector space, then $V$ has a basis.
  Any two bases for $V$ have the same number of vectors or all have infinite vectors.
\end{theorem}

\begin{definition} [Inner Product] \ \\
  An \textbf{inner product} on a vector space $V$ is a function
  $\langle \cdot, \cdot \rangle : V \times V \rightarrow \bb{R}$
  that satisfies the following properties for all $u, v, w \in V$ and $r \in \bb{R}$:
  \begin{itemize}
    \item $\langle u, v \rangle = \langle v, u \rangle$
    \item $\langle u + v, w \rangle = \langle u, w \rangle + \langle v, w \rangle$
    \item $\langle ru, v \rangle = r \langle u, v \rangle$
    \item $\langle u, u \rangle \ge 0$ and $\langle u, u \rangle = 0$ iff $u = 0$
  \end{itemize}
\end{definition}

For example, in $\bb{R}^n$, the standard inner product is defined as $\langle u, v \rangle = u_1 v_1 + u_2 v_2 + \dots + u_n v_n$.
In $R[x]$, we can set $\langle f, g \rangle = \int_{a}^{b} f(x)g(x)dx$.
The inner product allows us to define properties of a vector:

\begin{definition} [Length] \ \\
  If $V$ is a vector space with an inner product, then the \textbf{length} of a vector $v \in V$
  is defined as $|v| = \sqrt{\langle v, v \rangle}$.
\end{definition}

\begin{lemma} [Cauchy-Schwarz Inequality] \ \\
  If $u, v \in V$, then $|\langle u, v \rangle| \le |u| \cdot |v|$.
  Furthermore, iff $u$ and $v$ are linearly dependent, then $|\langle u, v \rangle| = |u| \cdot |v|$.
\end{lemma}

\begin{definition} [Angle Between Vectors] \ \\
  If $u, v \in V$, then the following holds: $\cos(\theta) = \frac{\langle u, v \rangle}{|u| \cdot |v|}$.
\end{definition}

\begin{definition} [Orthogonal Vectors] \ \\
  If $u, v \in V$, then $u$ and $v$ are \textbf{orthogonal} if $\langle u, v \rangle = 0$.
\end{definition}

\newpage

\section{Local Curve Theory}

\SecHead{Curves in $\bb{R}^3$}
We can think of curves as either a geometric set of points,
or as the path traced out by a particle moving in $\mathbb{R}^3$
as a function of a parameter, $t$.
The second method offers a means of apply calculus
to describe the geometric behavior of the curve.
The same geometric curve could have many parameterizations,
but the geometric properties do not depend on the parameterization

\SecHead{Basic Definitions}
We restrict the curves we study to the following constraints:
\begin{itemize}
  \item The curve should be described by a differentiable function
  \item {
        The derivative of the curve's parameterization
        shall not be equal to zero on the interval of interest
        }
\end{itemize}

Curves that fit these constraints are labeled as 'regular'.
We can formally define regularity as such:

\begin{definition}[Regular Curve]\ \\
  A \textbf{regular curve} in $\bb{R}^3$ is a function
  $\alpha: (a, b) \rightarrow \bb{R}^3$
  of class $C^k$ for some $k \ge 1$ and for which
  $\frac{d\alpha}{dt} \ne 0$ for all $t \in (a, b)$
\end{definition}

With our regular curve, $\alpha$, we can define other vector fields
around $\alpha$.
These include the \textbf{velocity}, \textbf{tangent}, and \textbf{binormal} fields.
Focusing on just the velocity field, we have:

\begin{definition}[Velocity, Velocity Vector Field, \& Speed] \ \\
  The \textbf{velocity vector} to a regular curve $\alpha(t)$
  is the derivative $\frac{d\alpha}{dt}$.
  The \textbf{velocity vector field} is the vector value function $\frac{d\alpha}{dt}.$
  The \textbf{speed} of $\alpha(t)$ is the length of the velocity vector,
  $|(\frac{d\alpha}{dt})|$
\end{definition}

\begin{definition}[Tangent Vector Field] \ \\
  The \textbf{tangent vector field} to a regular curve $\alpha(t)$
  is the vector valued function $T(t) = \frac{d\alpha/dt}{|d\alpha/dt|}$
\end{definition}

% ------------------------------------------------------------------------------
% Reference and Cited Works
% ------------------------------------------------------------------------------

\bibliographystyle{IEEEtran}
\bibliography{References.bib}

% ------------------------------------------------------------------------------

\end{document}

